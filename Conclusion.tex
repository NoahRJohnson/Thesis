\chapter*{Conclusion}

We have shown that the proposed vehicle design is capable of using wheel encoders and a smartphone to localize itself with respect to UTM coordinates.

%Project Limitations
However, this localization is less precise than it could be, due to several flaws in the rover's design. The differential drive model is a simple approximation to the skid steering rover, causing large errors in the yaw state variable to accumulate quickly in the wheel odometry. The covariance matrices for the sensors are hard-coded rather than dynamically computed, and need to at least be empirically calibrated. And the magnetometer readings are affected by the magnetic field generated by the DC motors. This effect could be alleviated by adding a second level on top of the rover, putting more distance between the smartphone and motors.

%Original goal:
%Create a cheap outdoors autonomous robot. Given a 'map' of the New College campus, this robot should be able to navigate from one outdoors location to another using footpaths. In doing so, it should dynamically avoid obstacles such as people, and recalculate alternate routes when a route is unexpectedly blocked.

%Possible Extensions

It is hoped that this project will be a jumping off point for future work. The logical next step would be to integrate the ultrasonic sensor's range and angle data into the project, as a slower approximation of LIDAR range data, which will allow the system to dynamically generate a map of the immediate area around the rover. Localization can then be performed in real-time with respect to this map.

%The most popular algorithm for global localization (i.e., localization including range data) is the adaptive Monte Carlo localization (amcl) system.


% not sure if this summary is accurate
%Amcl is another recursive Bayesian estimator based on Bayes Filter, which represents the state's belief distribution by a cloud of particles. Each particle represents a possible state, and each particle is assigned a probability weight based on the probability that the current sensor readings would be measured, given that that particle state was the true state. It uses random sampling to update its particle cloud over time.

%localization with respect to laser-based map landmarks 

%Range data would be fed into amcl, and it would adjust the 
%coming from the first EKF node which %outputs continuous local filtered odometry, 

%Its output would also be discontinuous, as recognition of a landmark previously seen will cause discrete jumps in amcl's position estimate.

Another possible extension would be to add video feed as a new source of sensor data, either via the laptop's webcam or the smartphone's built-in camera. The camera would need shock absorbers, or else the video frames would wobble. Perhaps video stabilization techniques could be used to compensate for this, in which case visual odometry could prove useful. The Arduino microcontroller board chosen in this design could be replaced with a larger model, or by a Raspberry Pi, which is a system on a chip that costs - at the time of this writing -roughly \$15 more than the Arduino board, but comes with additional features such as built-in WiFi and Bluetooth for short-range communication. 

On the software side, the differential\_drive package should eventually be replaced by the diff\_drive\_controller package, which performs the same function but integrates more naturally into the ROS navigation stack. The collection of software used in this project is available online at the following url: \url{https://github.com/NoahRJohnson/AutoRover}.

% gps and imu chip instead of phone, as far as software goes, mention the ROS navigation stack to facilitate actual autonomous navigation rather than just localization.

%Possible idea for simple navigation: GPS Waypoints. GPS markers form a linearized path travel in a straight line from one GPS node to another
% Could create waypoints using google maps, manually creating waypoints, and exporting to kml file, turning that into a csv file maybe?
%https://www.google.com/maps/d/edit?mid=1wKgTvF7i7Xdpi42cw1ffAOrGDVw&ll=27.38460862376806%2C-82.56151826455687&z=17



\addcontentsline{toc}{chapter}{Conclusion}