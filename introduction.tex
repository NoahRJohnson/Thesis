\chapter*{Introduction}

%intro to the concept of navigation
Navigation of a robot involves finding where the robot is with respect to its environment, calculating a path from where it currently is to where it would like to be, executing commands to follow that path, and reacting to dynamic obstacles along that path to avoid collision. While the rover design presented in this thesis is capable of performing all of these tasks, only the first has been implemented. This task is known as  localization, since it involves localizing the position and orientation of the rover within its local environment.

% lit review, current state of the field
In order for a robot to localize itself with respect to its environment, it must either already know its environment via a static map, or dynamically generate a map of its surroundings. This latter approach is known as the simultaneous localization and mapping (SLAM) problem, and the most common algorithms used to solve it involve the use of ranging sensors. Ranging sensors measure the distance between the robot and surrounding obstacles.

%Since a map with GPS coordinates is provided, this is not SLAM, but a simplified navigation problem. 
However, in this project's current implementation a static map was assumed to exist. Thus ranging sensors - though purchased - were not utilized. The static map used is the Universal Transverse Mercator (UTM) 2D projection of a coordinate system onto the Earth's surface. A Global Positioning System (GPS) receiver was used for noisy global position information, and localization was implemented with respect to this map. 

%Dead reckoning is used to create a local position estimate. That's what EKF does. This estimate is then updated based on the rover's position in comparison to nearby landmarks in the dynamically generated map.

%When you don't know where you are with respect to any global reference, i.e. you are a kidnapped robot dropped in a dark room with no idea where you are (there is no presupplied map), then the problem of discovering where you are while mapping your environment is known as simultaneous localization and mapping (SLAM).

% this project's place in the field and goal

%The ultimate goal of this project is to create a mobile robot capable of navigating the New College of Florida campus during the day, with limited financial resources. While that end goal has not yet been reached, appreciable progress in designing and constructing the rover and its associated control software has been made. Localization with respect to a static map was implemented, and integration of existing range data into dynamic map generation is the next logical step. It is the author's hope that further work with this hardware will be completed at New College of Florida.

% design and its importance (broader access to field)
The rover's design is carefully laid out through this thesis in the hope that others interested in experimenting with autonomous navigation will be able to replicate the construction. The cheap material cost may help ease the financial burden for those who desire hands-on experience in this field. The design consists of a four-wheeled skid-steering aluminum rover base, an Arduino microcontroller, and a motor driver. It is controlled by a laptop connected via USB. Sensors available include two motor rotary encoders, a mobile smartphone, and an ultrasonic distance sensor. The Arduino acts as a low-level robotic controller, sending wheel encoder and range data to the laptop, and accepting motor velocity commands. An Android app publishes Intertial Measurement Unit (IMU) and GPS data from the mobile phone, and the laptop fuses these readings with the wheel encoder data to form an estimate of pose and velocity using an extended Kalman filter.

% Choice of Sensors
The sensors available are limited in precision, but were chosen for their cheap cost and wide availability. Rather than purchasing separate IMU and GPS chips to interface with the Arduino microcontroller, we used the built-in sensors on a smartphone. These built-in sensors are less accurate than their external counterparts, but often present no additional cost to acquire for a college student.

Infrared (IR) sensors are cheap ranging sensors common in small robotics projects. However, this rover was designed to work outdoors during the day, and the ambient IR radiation from the sun makes these sensors unusable. Meanwhile light detection and ranging (LIDAR) sensors are highly expensive ranging sensors which are considered the industry standard for high resolution, high accuracy point cloud mapping in a 360 degree arc. Due to the financial constraints of this thesis, these range sensors are out of the question. So a third option was chosen: the ultrasonic sensor. Ultrasonic sensors are medium-distance ranging sensors which emit high-frequency sonic pulses and listen to the echoes. They are capable of detecting objects within a cone in front of them. They have reduced accuracy in rain or snow, and can suffer from confusing double echoes. However, they are nearly as inexpensive as IR sensors, and by panning back and forth, can slowly give a 180 degree arc distance measurement.

%Thesis Structure
The rest of this thesis is structured as follows.

Chapter 1 covers the probability theory behind the extended Kalman filter, which is an iterative algorithm used to fuse noisy sensor data into a local state estimate for the rover. It may be skipped if one is not interested in the mathematical details. Chapter 2 describes the hardware components used in this project, their electrical connections, and the general design. Chapter 3 continues the description of physical connections with respect to the Arduino circuit board, and also describes the software which runs on that board and how it interfaces with the rest of the system. Chapter 4 gives a brief overview of the Robot Operating System (ROS), and how it is used in this project. It then touches on the distinct software processes that make up the system. Chapter 5 gives the results of an experimental localization test run conducted with the rover.

\addcontentsline{toc}{chapter}{Introduction}