\chapter{Introduction}

\section{Goal}
Create a cheap outdoors autonomous robot. Given a 'map' of the New College campus, this robot should be able to navigate from one outdoors location to another using footpaths. In doing so, it should dynamically avoid obstacles such as people, and recalculate alternate routes when a route is unexpectedly blocked.

\section{Navigation}
intro to the concept of navigation

Since a map with GPS coordinates is provided, this is not Simultaneous Localization and Mapping (SLAM), but a simplified navigation problem. 

\section{Choice of Sensors}
IR, Microsoft Kinect
can't use because the rover will be outdoors during the day and the sun gives off ambient IR radiation.

LIDAR - state of the art

And LIDAR sensors would be too costly. constrained by cheap hardware, and the inability to use IR or LIDAR sensors.

\section{Thesis Structure}
This thesis is organized as follows.

Chapter 2 covers the probability theory behind the extended kalman filter, which is an iterative update algorithm used to fuse noisy sensor data into a local state estimate for the rover. It may be skipped if you aren't interested in the mathematical details.

Chapter 3 describes the hardware components used in this project, their connections, and other considerations of the design.

Chapter 4 continues the description of physical connections with respect to the Arduino board used, but begins to describe the software which runs on that board and how it interfaces with the rest of the system.

Chapter 5 gives a brief overview of the Robot Operating System (ROS), and how it's used in this project. It then meanders through every software process used, and how they communicate through ROS.

Chapter 6 finishes up with a brief mention of how far the project got, what more needs to be done to complete it, and limitations of the current design.